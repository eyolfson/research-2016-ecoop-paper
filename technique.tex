Our ConstSanitizer tool generates instrumented code which, when executed,
prints out notifications about \wstcqs
\footnote{We previously implemented a static analysis which was an unpublishable
          dead end. Most of its reported violations required calling context to
          make sense of; context-sensitive interprocedural analysis would thus
          be required to get meaningful results.
          Static counts of mutables and const casts were too overwhelming.
          Furthermore, imprecision due to pointers made its results unusable.}.
ConstSanitizer builds upon LLVM~\cite{llvm} and was inspired by existing
sanitizers including AddressSanitizer~\cite{2012-usenix-serebryany} and
MemorySanitizer~\cite{2015-cgo-stepanov}.

We implemented ConstSanitizer by extending the \texttt{clang} frontend and
adding instrumentation passes on \texttt{llvm} bitcode.
The instrumented code calls hooks in our modified version of \texttt{llvm}'s
\texttt{compiler-rt} runtime library.
Figure~\ref{figure:overview-src-to-obj} depicts our processes for compiling
instrumented code.
Plain text indicates inputs and outputs; outlined boxes indicate existing
software components; and light gray boxes indicate our modifications.

We first describe our modifications to the \texttt{clang} frontend.
When the developer enables ConstSanitizer (using a command-line flag), our
frontend adds metadata about initialization expression extents to the bitcode.
This metadata notifies the \texttt{llvm}-level instrumentation about
source-level constructs that would otherwise be lost in translation to bitcode.

Specifically, we modified \texttt{clang}'s bitcode generator at variable
declaration statements.
At statements of the form \code{type var = expr} we mark the instructions
making up \code{expr} so that our \texttt{llvm}-level
instrumentation can ignore them.
The rationale for ignoring those writes is that the primary user-visible write
from a \texttt{clang} declaration statement is to \code{var}, on the
left-hand side.
We empirically observed that other writes within \code{expr} are almost always
initialization writes to \code{var}, which ought
not to be reported even if \code{var} is \const{}.
Although a programmer may include explicit (side-effecting) writes within
\code{expr}, we ignore such writes to eliminate the false positives
that otherwise occur due to initialization writes.

\begin{figure}
  \centering
  \begin{tikzpicture}[->,>=stealth',shorten >=0.5mm,node distance=3cm]
    \node (src) {Source File(s)};
    \node (clang) [xshift=1.5cm, yshift=-0.5cm, draw, node distance=1cm,
                   below of=src]
          {\texttt{clang}};
    \node (flag) [fill=black!15, text width=2.5cm, text centered,
                  right of=src]
          {sanitizer flag};
    \node (clang-mod) [fill=black!15, text centered, text width=3cm,
                       right of=clang]
          {add metadata for expression extents};
    \node (llvm) [rectangle, draw, right of=clang-mod] {\texttt{llvm}};
    \node (llvm-mod) [fill=black!15, text centered, text width=3cm,
                      right of=llvm]
          {instrument LLVM bitcode to track \const{} and check stores};
    \node (obj) [node distance=1.8cm, below of=llvm-mod] {Object File(s)};

    \path (src) edge (clang)
          (flag) edge (clang)
          (clang) edge (clang-mod)
          (clang-mod) edge (llvm)
          (llvm) edge (llvm-mod)
          (llvm-mod) edge (obj);
    \node (clang2) [xshift=-1.5cm, yshift=-0.5cm, draw, node distance=1cm,
                   below of=obj]
          {\texttt{clang}};
    \node (flag) [fill=black!15, text width=2.5cm, text centered,
                  left of=obj]
          {sanitizer flag};
    \node (clang-mod) [yshift=1.5cm, fill=black!15, text centered,
                       text width=3cm, below of=clang2]
          {link with modified runtime library};
    \node (compiler-rt-mod) [xshift=-1.5cm,fill=black!15, text centered,
                             text width=5cm, node distance=3.7cm,
                             left of=clang-mod]
          {includes code to report \wstc{} at runtime};
    \node (compiler-rt) [rectangle, draw, node distance=1.5cm,
                         above of=compiler-rt-mod]
          {\texttt{compiler-rt}};
    \node (exe) [xshift=-1.5cm, node distance=1.5cm, below of=clang-mod]
          {Executable};

    \path (obj) edge (clang2)
          (flag) edge (clang2)
          (clang2) edge (clang-mod)
          (clang-mod) edge (exe)
    (compiler-rt) edge (compiler-rt-mod)
    (exe) edge (compiler-rt-mod);
  \end{tikzpicture}
  \caption{ConstSanitizer generates instrumented LLVM bitcode which reports
           writes through \const{} qualifiers at runtime.}
  \label{figure:overview-src-to-obj}
\end{figure}

We use \const{}-ness information as provided by declarations,
rather than implementing a taint-based approach.
Listing~\ref{listing-false-negative-example} shows a false negative caused by
our approach.
The debugging information for \texttt{y} gives shadow value
$(\texttt{00})_2$.
Hence, on the write through \texttt{y}, we do not report a \wtc{},
because we do not propagate \const{}-ness information from variable
initializers.
One might expect this write to trigger a report since \texttt{y} aliases
the read-only reference \texttt{x}.
(We report a write if the cast is part of a function argument.)

\begin{listing}[ht]
  \caption{C++ source code showing a false negative due to expression handling.}
  \label{listing-false-negative-example}
  \centering
  \includestandalone{listing/false-negative-example}
\end{listing}

Most of our instrumentation lives in a custom \texttt{llvm} pass that
generates code to track \const{}-ness of the program's values.
The instrumentation manipulates shadow values to track \const{} qualifiers at
every instruction that generates a pointer value.
The \const{} information relies on type tables from DWARF 4 debugging
information.
In \texttt{llvm}, this includes all variables in functions---all local
variables are allocated on the stack and are pointers.

Our dynamic analysis returns (as one might expect) no false positives, since it
observes program executions.
However, it depends on the accuracy of the debugging information and
metadata, which it uses to identify which variables are \const{}-qualified in
the source and to identify initialization expression extents.
We ran into one false positive in our results which we believe is the result of
the metadata being invalidated between LLVM passes.
Throughout the remainder of the section, we point out a couple of cases where
our analysis must approximate intended \const{}-ness because actual
\const{}-ness information does not exist.

\begin{table}[!b]
  \caption{Shadow values encode available \const{}-ness restrictions on
           variables.}
  \label{tab:shadow-value-example}
  \centering
  \begin{tabular}{@{}lrlr@{}} % \toprule
    \renewcommand{\arraystretch}{1.2}
    \textbf{Declaration} & \textbf{Shadow value} & \textbf{Example statement}
    & \textbf{Allowed} \\ \midrule
    \code{int x} & $(0)_2$ & \code{x = 5} & $\checkmark$ \\
    \arrayrulecolor{black!10} \hline
    \code{\const{} int x} & $(1)_2$ & \code{x = 5} & $\times$ \\
    \arrayrulecolor{black!30} \hline

    \code{int * x}  & $(00)_2$ & \code{x = y} & $\checkmark$ \\
    && \code{*x = 55} & $\checkmark$ \\ \arrayrulecolor{black!10} \hline

    \code{int *\const{} x} & $(01)_2$ & \code{x = y} & $\times$ \\
    && \code{*x = 55} & $\checkmark$ \\ \hline

    \code{\const{} int * x} & $(10)_2$ & \code{x = y} & $\checkmark$ \\
    (or \code{int \const{} * x}) && \code{*x = 55} & $\times$ \\ \hline

    \code{\const{} int *\const{}} & $(11)_2$ & \code{x = y} & $\times$ \\
    (or \code{int \const{}  *\const{}}) && \code{*x = 55} & $\times$  \\
    \arrayrulecolor{black}
  \end{tabular}
\end{table}

\subparagraph*{Structure of shadow values.}

A shadow value consists of $n$ bits tracking \const{}-ness (where $n$ is the word length of
the processor architecture).
Each bit represents whether a pointer or pointee has a \const{}
qualifier or not.
The rightmost bit represents the \const{} qualifier of the value itself.
Bits to the left (if the value is a pointer) represent what the pointer
transitively points to.
Our encoding supports pointers up to $n-1$ levels deep on $n$-bit processors
(64 for our experiments).
Figure~\ref{figure:shadow-values} depicted our encoding of shadow values, while
Table~\ref{tab:shadow-value-example} shows how shadow values represent
sample \const{}-ness settings and corresponding writes allowed.

\subparagraph*{Shadow value computation.}

We next describe how we create and propagate shadow values.
Our ConstSanitizer instrumentation dynamically propagates shadow values
representing \const{} qualifiers through a program's instructions.
Our goal is to monitor 1) writes to \texttt{mutable} fields; 2) locations where
\const{} has been cast away; and 3) transitive writes, to pointees of fields,
through \const{} references.
Table~\ref{tab:technique-summary} summarizes the analysis rules.

\begin{table}[!b]
  \caption{Dynamic analysis rules showing computation of shadow value for result
           \texttt{\%1}.}
  \label{tab:technique-summary}
  \centering
    \small
  \begin{tabular}{l p{8cm}}
    \textbf{Instruction} & \textbf{New shadow value} \\
    \midrule
    \texttt{\%1 = alloca ...} &
      from \const{} qualifiers in debugging
      information, consistent with Figure~\ref{figure:shadow-values}.
      \\[0.5em]
    \texttt{\%1 = getelementptr \%2} &
      by logically shifting left \texttt{\%2}'s shadow value once for each
      dereference this instruction represents.

      \textbf{if field access}: check \const{} qualifier of base object; for
      (immutable) \const{} base objects, new shadow value is all ones, otherwise all zeros.
      \\[0.5em]
    \texttt{\%1 = call(\%2)} &
      loaded from return shadow value in Thread-Local Storage (TLS).

      \textbf{for pointer arguments \texttt{\%2}:} also write shadow values to
      appropriate TLS slots for the function call; if the call and argument are
      marked as ignored, write all zeros for the shadow value for the argument.
      \\[0.5em]
    \texttt{\%1 = phi}/\texttt{select} \texttt{...} &
      carry out same operation on shadow value operands.
      \\[0.5em]
    \texttt{\%1 = bitcast \%2} &
      from the shadow value for \texttt{\%2}, if compatible; \\ & otherwise all
      zeros.
      \\[0.5em]
    \texttt{\%1 = load \%2} &
      logical shift right of \texttt{\%2}'s shadow value.
      \\[0.5em]
    \texttt{store \%2, \%1} &
      check rightmost bit of shadow value for \texttt{\%1}, report \wtc{} if
      set. (Only applies if the instruction not ignored as an initializer.)

      \textbf{if \texttt{\%2} is function argument:} load shadow value for
      \texttt{\%2} from TLS, left shifted once. New shadow value is bitwise OR
      of shifted value with previously computed shadow value for \texttt{\%1}.

      \textbf{if \texttt{\%2} is ``this'' function argument:} same steps as
      above, except skip the bitwise OR step.

      \textbf{if \texttt{\%2} is ``this'' function argument for destructor:} shadow
      value for \texttt{\%1} is $(00)_2$.
      \\[0.5em]
    \texttt{\%1 = extractelement}
       &
       all zeros.\\
    \texttt{\%1 = extractvalue}\\
    \texttt{\%1 = inttoptr} \\
    \texttt{\%1 = landingpad}
      \newline
      \\
  \end{tabular}
\end{table}

\texttt{llvm} bitcode uses \texttt{alloca} instructions to introduce new pointer
values.
Our \texttt{llvm} pass instruments each \texttt{alloca} instruction with the
appropriate shadow value, as extracted from the type information in the source
code, using standard \texttt{clang} debug information.

Ultimately, our instrumentation verifies the behaviour of \texttt{store}
instructions.
Recall that we exclude \texttt{store} instructions that come from the right-hand
side of a declaration statement.
For all other \texttt{store}s, we check whether the operand---the location being
written-to---represents a \const{}-qualified type.
The rightmost bit of the shadow value provides this information.
If that bit is 1, an execution of this \texttt{store} instruction is a write to
a \const{}-qualified location.
We insert a call to our runtime library to check the value of the bit and to
report a \wtc{} if the bit is 1.
(We later discuss a special case for store instructions where the value being
stored is a function argument.)

Conversely, \texttt{load} instructions return a pointer that represents a single
pointer dereference.
To compute the returned shadow value, we right shift the operand's shadow value.

\texttt{llvm}'s \texttt{getelementptr} (GEP) instruction accesses arrays and
fields of objects.
This instruction preserves type safety through dereferences in the compilation
process and is a safe alternative to directly generating pointer arithmetic
code.
Our instrumentation performs a logical shift right by one bit for every pointer
dereference implied in the GEP instruction.
Our treatment of GEP implicitly handles transitive immutability as follows: when
a GEP accesses an object field, and the containing object is \const{}-qualified,
we generate a shadow value as if the field had a \const{} qualifer on every type
for the contained field.
This treatment implies checks for transitive immutability; generating a
non-\const{} shadow value here would generate the same bitwise immutability
checks for \const{} as specified by the C++ standard.

Our instrumentation propagates \const{}-ness information (in
shadow values) alongside references to that location.
In C++, access restrictions to a location depend on whether the
program is accessing that location through a \const{} reference or not.
Therefore, in the presence of casts and pointer arithmetic, there is no ground
truth about the \const{}-ness of the resulting references and we must make a
reasonable under-approximation as to \const{}-ness.

We next discuss casting-related \texttt{llvm} instructions.
The \texttt{bitcast} instruction converts a value into a specified type.
If a program converts a pointer between equally-indirected pointer types, then
we copy the old shadow value to the result.
(C++ \const{}-casts do not appear at LLVM bitcode level, nor
do the component of a C-style cast that manipulates \const{}-ness.
ConstSanitizer preserves declared \const{}-ness for such variables.)
Otherwise, we choose to assume that the instruction's result has no \const{}
qualifiers.
We make this assumption in all cases for the \texttt{inttoptr} instruction,
which represents pointer arithmetic not handled by the GEP instruction, as well
as for \texttt{extractvalue} and \texttt{extractelement}.

Our instrumentation stores shadow values for function calls' arguments and
return values using thread local storage (TLS).
In the straightforward case, we store shadow values for pointer arguments in
TLS slots reserved for each argument.
However, we ignore pointer arguments' \const{} qualifiers if the call and the
argument are both part of a variable declaration.
As for (pointer) return values: if a function was instrumented by our tool, then
we read the shadow value from the appropriate TLS slot.
We also store a mutable shadow value in the return value TLS slot, in case the
function had not been instrumented by our tool.
Our instrumentation either reads the approximation or, if applicable, the actual
return value generated by the called function.
In the presence of callbacks from uninstrumented code back to instrumented code,
our instrumentation may use stale shadow values and report extraneous results
based on these stale shadow values.

We have a special case for store instructions where the value stored is a
function argument, as mentioned above.
Consider a store instruction ``\texttt{store value, location}''.
We compute the shadow value for ``\texttt{location}'' as follows.
First, we get the shadow value for ``\texttt{value}'' from the TLS.
Then, we adjust this shadow value to be compatible with the type of
``\texttt{location}'': our encoding requires one logical shift to match the type
of ``\texttt{value}'' to that of ``\texttt{location}''.
We bitwise OR the shadow value for ``\texttt{location}'' previously computed
(from an \texttt{alloca} instruction) with the shifted value to get the new
shadow value for ``\texttt{location}''.
This preserves \const{} qualifiers of the original argument and of the local
variables in the function.

There are two further special sub-cases for store instructions and function
arguments for (i) method and (ii) destructor calls.
Listing~\ref{listing-llvm-example} illustrates sub-case (i).
Here, \texttt{foo()} is declared \const{}.
The compiler will hence treat \texttt{this} as \const{} within \texttt{foo()}.
However, for our dynamic analysis, we want to detect writes based on the
\const{}-ness of \texttt{this} from the caller; in method \texttt{bar()} in
Listing~\ref{listing-llvm-example}, receiver object \texttt{nc} for
\texttt{foo} is not \const{}, so we do not want to report the call's (transitive) store
to \texttt{x}.
\texttt{foo}'s method arguments appear as ``\texttt{value}'' operands of
\texttt{store} instructions while the ``\texttt{location}'' is an
\texttt{alloca} within the function.
We set the shadow value of the associated \texttt{alloca} instruction to the
value of the argument after applying a logical shift left by one (since it's a
pointer).
This treatment properly ignores \const{} qualifiers added due to callee
method signatures.

For sub-case (ii), destructors, we do not want to report any writes through
\texttt{this} as the object no longer exists after the call (so that writes to
the object aren't visible in any case).
We handle this case by simply assuming that the \texttt{this} argument is
mutable.
For all other arguments, we do a bitwise OR between the \texttt{alloca} shadow
value and the argument shadow value logically shifted left by one, which
maintains all \const{} qualifiers.

\begin{listing}[ht]
  \caption{C++ source code showing calls to method \texttt{foo()} (with
           its definition and associated LLVM bitcode) from \const{} context
           \texttt{cc} and non-\const{} context \texttt{nc}.}
  \label{listing-llvm-example}
  \centering
  \includestandalone{listing/llvm-example}
\end{listing}

\subparagraph*{Shadow value computation example.}

Listing~\ref{listing-llvm-example} presents the C++ source code for
\code{C::foo}, a \const{} qualified method, and the associated LLVM bitcode.
Consider the \code{bar} function, which calls \code{foo} twice, first with
mutable (i.e. non-\const{}) receiver object \texttt{nc} and then with \const{}
receiver object \texttt{cc}.
Within \code{bar}, the shadow value of \texttt{nc} is $(0)_2$ and the shadow
value of \texttt{cc} is $(1)_2$.
Our instrumentation assigns shadow values for each LLVM instruction with a
pointer result. We instrument \code{C::foo} as follows:

\begin{itemize}
  \item The first instruction, \texttt{alloca}, stores its result in
        \texttt{\%1}.
        Since it is an \texttt{alloca} instruction, we obtain its shadow value
        from \texttt{clang} debugging information.
        The associated shadow value is $(10)_2$: in this \const{}-qualified
        method, the type of \code{this} is that of a \const{} pointer to the
        containing class, \code{\const{} C *}.
  \item At the \texttt{store} instruction, without special handling, we would
        load the shadow value of argument \texttt{\%this} from the TLS;
        logically shift left the shadow value by one to account for the fact
        that we are performing a \texttt{store} to memory allocated for that
        argument; and bitwise OR the resulting shadow value with the original
        shadow value for \texttt{\%1}.
        In our example, whether the receiver object is \texttt{cc} or
        \texttt{nc}, the shadow value for \texttt{\%1} is $(10)_2$.
  \item Next, we obtain the shadow value for the result of the \texttt{load},
        \texttt{\%2}.
        As \texttt{\%2} returns a pointer, we shift \texttt{\%1}'s (the
        operand's) shadow value right by one, giving a shadow value of $(1)_2$.
  \item Next, the \texttt{getelementptr} instruction results in a pointer to the
        class's \texttt{x} field.
        Our instrumentation of \texttt{getelementptr} could produce two
        different shadow values, depending on the instruction's operand.
        In this case, \texttt{\%2} is a \const{} object, and the resulting
        shadow value for a fully \const{}-qualified \texttt{x} field is
        $(11)_2$.
  \item Next, we obtain the shadow value for the \texttt{load} result
        \texttt{\%4} using the same technique as for \texttt{\%2}.
        The resulting shadow value is $(1)_2$.
  \item Finally, we insert a check at the \texttt{store} instruction.
        In this case, the least significant bit of the shadow value associated
        with location (\texttt{\%4}) is 1.
        Therefore we would dynamically report a \wtc{} at the write to field
        \texttt{x} of the \const{} method.
\end{itemize}

For methods, this instrumentation is not enough.
We only want to report a \wtc{} for the call with \const{} receiver object even
though both objects call the same static method.
Before the call to \code{foo}, our instrumentation stores the shadow value of
the receiver object in its TLS slot.
Our instrumentation of \code{foo} looks for \texttt{store} instructions that use
the receiver object and recomputes the shadow value of the location.
Here, we load the shadow value from its TLS slot and shift left
to match the type of the expected shadow value of \texttt{\%1}.
For \texttt{nc}'s call, this shadow value is $(00)_2$.
Since \code{foo} is a method, we ignore the original shadow value of
\texttt{\%1} ($(10)_2$) and overwrite it with new shadow value
$(00)_2$.
Following the remaining steps in \code{foo} as above, the shadow value of
\texttt{\%4} is now $(0)_2$ and we do not report a \wtc{}.
In the \texttt{cc} case, we would follow the same steps, but instead report a
\wtc{}, because the shadow value would be $(10)_2$.
