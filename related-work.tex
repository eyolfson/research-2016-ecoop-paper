We discuss a number of areas of related work: immutability (and related
approaches) for Java; purity analyses; and dynamic analyses that inspired our
approach.
The Java programming language has no exact analog to C++'s
\const{} operator.
Related work defined immutability annotations for Java and statically and
dynamically verified that programs satisfy their annotations.
Potanin et al~\cite{lncs-2013-7850-potanin} provide a recent discussion of
immutability terminology and compare research implementations in depth.
(Our implementation has at its core a dynamic analysis verifying that C++ programs
satisfy a strengthened version of their \const{} annotations.)
A different stream of related work verifies whether (Java) methods are pure,
i.e. have no visible side-effects; there is a strong connection between purity
and immutability.

\subparagraph*{\const{} for Java, and similar projects.}

Javari~\cite{meng-tschantz, oopsla-2005-p211-tschantz} allows its users to
specify read-only references in Java programs.
It aims to ensure that a \texttt{readonly} (effectively, deeply-immutable
\const{}) typed object does not mutate its state or any state transitively
reachable through its references.
Javari, like C++, includes a \texttt{mutable} keyword, which allows developers
to specify that a field may be modified regardless of \texttt{readonly}
qualifiers.
Javari also inserts dynamic checks to verify that downcasts maintain the
immutability qualifier of the type.
Essentially, Javari provides a safer version of C++'s \const{} that, due to the
nature of Java, maintains deep immutability unless the developer explicitly opts
out of the checks.

Our work provides similar guarantees to those provided by Javari.
Since Javari does not have an underlying C++ \const{} specification to build on
top of, it has to implement all of those checks itself.
In terms of what we check, our work matches Javari in terms of transitivity and
downcasts.
Our work also reports writes to mutable fields at runtime.

Unlike Javari, we could investigate the behaviour of a set of real-world
benchmark programs, developed against the C++ \const{} semantics (and which, by
necessity, satisfy those semantics).
Our empirical study therefore points out the difference in practice between
\const{} semantics as they exist---shallow immutability, mutable, and
\const{} casting; and a stronger version of these semantics---deep
immutability, no mutable, and no const casting.

A related concept to \const{} is that of stationary fields, as proposed by Unkel
and Lam~\cite{popl-2008-p183-unkel}.
A stationary field is similar to a Java \texttt{final} field.
A Java \texttt{final} field can only be written to in a constructor.
By contrast, a stationary field is only written to before it is read.
In essence, a stationary field acts like a \texttt{final} field but with fewer
restrictions on where initialization may occur.
Nelson et al~\cite{NelsonPearceNoble12ProfilingFieldInitialisationJava}
performed a follow-up study using a dynamic analysis which determined that
72--82\% of fields in Java programs are stationary.
Their work, like ours, empirically explores how programs use (or could use, in
their case) language features.

\subparagraph*{Purity analyses.}

A pure method does not modify any state accessible before the
method was called.
Pure methods may create and modify objects to return to the caller.
A function which writes to no global state and has all arguments transitively
\const{} is pure.

%% JPure~\cite{cc-2011-p104-pearce} is a purity analysis for Java that leverages
%% method annotations to mark methods as pure.
%% A modular verification approach on the annotations enables their analysis to
%% scale; JPure uses an intra-procedural analysis.
%% If a method has a pure annotation they ensure that it does not write to any
%% fields, call any methods not marked as pure, or override a method that is not
%% also marked as pure.
%% Fresh values (newly declared) and fields that act as local variables (only
%% accessed in a certain method) are also allowed.
%% They found that over 40\% of methods could be marked as pure for the Java
%% standard library.

ReIm and its corresponding type inference system,
ReImInfer~\cite{oopsla-2012-p879-huang}, is similar to Javari, except that its
type system is context sensitive.
ReIm %does not provide a way to exclude (mutable) fields from being considered for immutability and 
was designed to find method purity.
Its type system allows the immutability of the
return type to match that of the calling reference.
This allows methods to be reused without requiring mutable and
read-only versions.
ReImInfer is a type inference system that maximizes the amount of methods
marked as \texttt{readonly}.
They report that 41--69\% of methods can be marked as \texttt{readonly}.

Other systems also verify method purity.
JPPA is a combined pointer and escape analysis for
Java~\cite{vmcai-2005-p199-salcianu, phd-2006-salcianu}.
S\unichar{259}lcianu and Rinard found over half the methods they analyzed were pure.
Rytz et al~\cite{ftfjp-2013-rytz} instead use a simpler flow-insensitive
analysis and find similar results.

Artzi et al proposed a combined dynamic and static analysis for
mutability~\cite{ase-2007-p104-artzi}.
Their analysis determines the mutability of method parameters.
The goal of that work was to scale better and produce more precise results than
static analysis alone.

\subparagraph*{\const{}-correctness through abstract machines.}

Chisnall et al~\cite{asplos-2015-chisnall-p117} propose a memory-safe abstract
machine for C which can be used to verify that immutable objects
(declared with \const{}) are never mutated through non-\const{} aliases.
This resembles our approach of using shadow values to track the declared
\const{}-ness of an object.
(They do not investigate transitive immutability.)
Our study, by contrast, focuses solely on \wstc.
All of their subject programs removed a \const{} qualifier at
some point.
It was beyond the scope of their work to investigate how and why their subject programs
removed the \const{} qualifier.

\subparagraph*{Other dynamic analyses.}

Our dynamic analysis approach is similar to approaches used in
Umbra~\cite{2010-cc-zhao} and Dr.~Memory~\cite{2011-cgo-bruening}.
Like Umbra and Dr.~Memory, we use shadow values to detect interesting
program behaviours.
However, ConstSanitizer builds directly on LLVM and does not use a
dynamic instrumentation platform.
Furthermore, the properties that we verify are novel \const{}-related
properties, compared to Dr.~Memory, which looks for memory errors such
as accesses to unallocated space, and Umbra, which helps developers understand
threads' memory access patterns and implements almost-free custom watchpoints.
