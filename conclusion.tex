The \const{} qualifier in C++ is extensively used in real-world code,
but developer intent behind \const{} usage is unclear.
In this paper, we have presented our ConstSanitizer system.
ConstSanitizer dynamically detects writes through \const{} qualifiers which are
legal in C++ but which modify state transitively starting from a \const{}
qualifier, write to mutable fields, or write to values whose \const{}-ness has
been cast away.
Our results show that, although writes through \const{} are ubiquitous across
our 7 C++ and 1 C benchmark programs, there are only a small number (17) of
archetypes these writes.
We used our results to develop a classification of writes according to root
cause (transitivity, mutable field, or const cast) and attributes (synchronized,
not visible, buffer/cache, delayed initialization, incorrect).
Our work helps understand how the \const{} qualifier is used in practice and
leads us to conclude:

\begin{itemize}
  \item Developers definitely violate bitwise \const{} ({\bf RQ1}).
  \item The majority of \wtc{} archetypes (9/17) are incorrect code
        which observably change an object's state.
  \item On our benchmarks, programs \wtc{} about equally often using transitive
        writes through fields (8/17) and writes to mutable fields (8/17) ({\bf RQ2}).
  \item We observed four classes (N, B, D, S, discussed in
        Section~\ref{section:classification}) of valid reasons for \wstcqs{}.
        For instance, sometimes developers \wtc{} to
        delay initialization or implement buffer caches.
        Such writes should be automatically validated by yet-to-be-developed
        tools.
  \item About half (9/17) of the observed usages are invalid, consisting of methods
        which implemented exceptions to an object's const-ness; perhaps the developers
        chose to add one exception rather than remove const completely. ({\bf RQ3})
\end{itemize}

\subparagraph*{Acknowledgements.}
This work was supported in part by Canada's Natural Science and Engineering Research
Council as well as a Google Faculty Research Award.
