We continue with an example of a \const{} usage that must be accepted by C++
compilers~\cite{programming-langauges-cpp-n3690} but leads to undefined
behaviour when used with the C++11 standard library specification.

Listing~\ref{listing-motivating-example} contains function \texttt{writeId()},
which modifies field \texttt{id} of its parameter.
Function \texttt{evil()} takes a \const{}-qualified parameter, casts away the
\const{} qualifier, and calls \texttt{writeId()}.
Both these functions must be accepted by C++ compilers.

\begin{listing}[!htb]
  \caption{Method \texttt{evil()} violates the spirit of \const{} by causing a
           write to an externally-visible field of \const{} object
           ``\texttt{a}''.
           Circled numbers used for subsequent explanations.}
  \label{listing-motivating-example}
  \centering
  \includestandalone{listing/motivating-example}
\end{listing}

Function \texttt{writeId()} does not perform anything unexpected.
The write to \texttt{id} is legitimate with respect to \const{}:
\texttt{writeId()} has a non-\const{} reference to data and is therefore
entitled to write to it.
(\texttt{writeId()} conflicts with a different part of the library
specification---one oughtn't write to fields used as hashes---but that
conflict is beyond the scope of this work.)

Function \texttt{evil()} accepts a \const{} parameter ``\texttt{a}''; the
\const{} qualifier intuitively suggests that the state of ``\texttt{a}'' should
not change across a call to \texttt{evil()}.
(Section~\ref{section:background} explains the C++ semantics of \const{} in more
detail; Section~\ref{section:technique} explains the writes that ConstSanitizer
monitors.)
\texttt{evil()} then casts away the \const{} qualifier and calls
\texttt{writeId()}, which writes to the \texttt{id} field, thus changing the
state of ``\texttt{a}''.

Listing~\ref{listing-motivating-example} also contains client code.
This client code provides a \texttt{hash()} function for when objects of type
``\texttt{A}'' are used with the standard library.
It continues with a declaration of an \texttt{std::unordered\_map} \texttt{m},
which gets a \const{}-qualified object ``\texttt{a}'' added to it.
(In C++, objects like ``\texttt{a}'' and \texttt{m} are constructed upon declaration.)
Between the \texttt{insert()} call and the \texttt{find()} call, the program
calls \texttt{evil()}, which changes the \texttt{id} field, thus causing the
\texttt{hash()} of ``\texttt{a}'' to change.
As a result, the \texttt{find()} call may unexpectedly return \texttt{m.end()},
indicating that it did not find ``\texttt{a}'' in the expected bucket; that
result should be a surprise to the client.

In our example, the client code is doing nothing wrong, yet it may get an
unexpected result from the map.
The client should be entitled to believe that its \const{}-qualified
``\texttt{a}'' object does not change between the two map calls and that
the object is still in the map.

\begin{figure}[!htb]
  \centering
  \begin{tikzpicture}[->,>=stealth',shorten >=0.5mm,node distance=2cm]

    \draw [decorate,decoration={brace,amplitude=3mm},-,xshift=0pt,yshift=0pt]
          (-0.7,0.3) -- (8.5,0.3)node [black,midway,yshift=0.6cm]
          {\textit{Shadow Value}};
    \node (b1) {$b_{n+1}$};
    \node [right of=b1] (b2) {$b_n$};
    \node [right of=b2] (b3) {$b_{n-1}$};
    \node [right of=b3] (b4) {...};
    \node [right of=b4] (b5) {$b_0$};
    \node [below of=b1, node distance=1cm] (b1t)
          {Is \code{T} \const{}?};
    \node [below of=b2] (b2t)
          {Is \code{T *} \const{}?};
    \node [below of=b3, node distance=1cm] (b3t)
          {Is \code{T **} \const{}?};
    \node [below of=b5, node distance=1cm] (b5t)
          {Is \code{T $(\texttt{*})^n$} \const{}?};

    \path (b1t) edge (b1)
          (b2t) edge (b2)
          (b3t) edge (b3)
          (b5t) edge (b5);
  \end{tikzpicture}
  \caption{Our shadow values encode the \const{}-ness of each level of an $n$
           level pointer.}
  \label{figure:shadow-values}
\end{figure}

\subparagraph*{Analysis of example.}
We informally describe how ConstSanitizer works on our example.
Our LLVM-based tool actually operates at the \texttt{llvm} bitcode
level (assisted by metadata from \texttt{clang}), but for clarity we describe
shadow values and the effects of statements on them using C++ code.
Section~\ref{section:technique} describes our analysis as-implemented on top of
LLVM.

ConstSanitizer associates a shadow value with each variable.
This shadow value describes whether or not the variable, and each of its
dereferences, may be written to; Figure~\ref{figure:shadow-values} shows how
shadow values encode \const{}-ness.
We present the semantics of our shadow encoding more thoroughly in
Table~\ref{tab:shadow-value-example}.
ConstSanitizer propagates shadow values through the program's execution.
At each write, ConstSanitizer verifies that the shadow value permits a write,
and indicates a \wtcq{} if not.

We next show how ConstSanitizer works; circled numbers refer back to
Listing~\ref{listing-motivating-example}.

\begin{enumerate}[label=\protect\circled{\arabic*}]
  \item Program allocates \const{}-qualified new object ``\texttt{a}''.
        Using debug information, ConstSanitizer finds the \const{} qualifier in
        ``\texttt{a}''{}'s type, and associates shadow value $(1)_2$ with
        ``\texttt{a}'', indicating that ``\texttt{a}'' is \const{}.
  \item ConstSanitizer then propagates shadow value $(1)_2$ to the function call
        \texttt{evil()}.
        Inside function \texttt{evil()}, variable ``\texttt{a}'' is a reference,
        so we shift the shadow value to the left and obtain $(10)_2$ inside the
        function call boundary.
  \item Program casts from type \texttt{const A*} to type \texttt{A*}.
        Because ``\texttt{a}'' is a reference inside \texttt{evil()}, the
        address-of operation has no effect on the shadow value.
        (On a non-reference, taking the address would also result in a logical
        shift left of the shadow value.)

        The cast of the \const{} qualifier is invisible to LLVM, so
        ConstSanitizer propagates shadow value $(10)_2$ across the cast.
        (Had ``\texttt{a}'' originally not been \const{}, a pointer to it would
        have shadow value $(00)_2$ and it would have shadow value $(0)_2$.)

        Finally, ConstSanitizer passes shadow value $(10)_2$ for parameter
        ``\texttt{pa}'' of \texttt{writeId()}.
  \item Inside function \texttt{writeId()}, the instrumented write to field
        ``\texttt{id}'' observes that the shadow value of the address of the
        containing object is $(10)_2$.
        Because the left-hand side uses the \verb+->+ operator, ConstSanitizer
        shifts the shadow value back right once, giving shadow value $(1)_2$ for
        the containing object.
        Because the containing object is \const{}, we also apply a shadow value
        of $(1)_2$ to all writes to fields of that object.

        When executing the write, ConstSanitizer checks the right-most bit of
        the shadow value for the destination.
        Since this value is $(1)_2$, the program is writing a value through a
        \const{} reference.
        ConstSanitizer therefore signals a \wtc{}.
\end{enumerate}

\noindent Section~\ref{section:classification} contains our classification of \wstc{}.
We classify this write as root cause ``C'', a write after casting
away a \const{}, and assign attribute ``I'', for incorrect.
