Immutability is an important concept that simplifies reasoning about programs
and eases software maintenance.
Most importantly, immutability circumscribes possible side effects, 
so that (in some cases) a user of a function may avoid closely
examining the implementation of the function and its callees.
One concrete application of immutability is: if a developer knows that a library
function does not modify one of its arguments (including transitive arguments),
then they know that it is safe to call that library function with that argument
from multiple threads, as the function only requires read access to its
argument.

C++~\cite{programming-langauges-cpp-n3690} is a popular language that allows
programers to specify immutability using \const{}\footnote{Some of our
discussion also applies to C, but we focus
on C++ in this paper.}.
C++ experts such as Meyers recommend judicious use of
``\const{}-correctness''~\cite{effective-cpp} in C++ codebases.
It is generally clear when developers could use \const{}, but not (in
non-obvious cases) when they should use \const{}.

We distinguish between two uses of C++'s \const{} qualifier:
\const{}-qualified global/stack objects, whose data may never change (i.e.
immutable objects), and \const{}-qualified references/pointers to objects (i.e.
read-only references~\cite{lncs-2013-7850-potanin}).
(For now, assume that there are no casts that remove \const{} qualifiers and no
\texttt{mutable} storage class specifiers; these language features violate
immutability.)
An \emph{immutable object}'s fields may never change, and mutable references to
that object may never be created.
A \emph{read-only reference}, on the other hand, only guarantees immutability
for accesses through qualified references.
An object with a read-only reference to it may still be mutated through other,
mutable, references.
C++ enforces a shallow immutability guarantee (also known as bitwise \const{})
for writes through the read-only reference: while it is illegal to reassign the
fields of such an object, any referents of fields may change.
We expand on this in
Section~\ref{section:background}.

C++'s type system admits several workarounds to \const{}'s supposed
immutability guarantees.
Much research defines type qualifiers similar to C++ \const{}, but with stronger
guarantees.
These type systems do not have any holes in the type system, such as
unsafe casts.
Furthermore, they not only ensure that the field values do not change, but also ensure
that objects referenced through fields also do not change (i.e. deep, or
transitive, immutability; again, see Section~\ref{section:background} for more
details).

On the other hand, our industrial contacts have indicated that, in their
codebases, \const{} has been used to deter new developers from modifying certain
variables.
Such variables may be modified by an experienced developer ready to assume the
consequences~\cite{fang:_person}.
In such cases, \const{} serves an advisory role, but does not provide any
guarantees.

Our goal is to explore the space of possible meanings for immutability
declarations in C++ and to examine what guarantees developers appear to be
expecting in practice.
We developed a tool, ConstSanitizer, that instruments programs to identify
source locations that modify \const{}-qualified objects, using more restrictive
semantics than guaranteed by C++.
ConstSanitizer monitors \wstc{}, i.e. writes performed on
\const{}-qualified objects or references, either transitively (which is allowed
in C++), or through C++'s \const{} escape hatches.
To better understand \const{} usage in practice, we ran ConstSanitizer on a
benchmark suite and manually classified all \wstc{}.

Specifically, the goal of this work is to answer the following research
questions:
\vspace*{-0.25em}
\begin{enumerate}
  \item[({\bf RQ1})] Do developers perform shallow and transitive \wstc{}? \\[0.2em]
        \hspace*{1em}
        \begin{minipage}{4.5in} (Answer: Yes to both.) \end{minipage}~\\[-1em]

  \item[({\bf RQ2})] How do developers \wtc{}? \\[0.2em]
        \hspace*{1em}
        \begin{minipage}{4.5in}
          (Answer: By directly writing to fields of \const{}-qualified objects
          and through transitive writes; both are about equally common.)
        \end{minipage}~\\[-0.5em]
  \item[({\bf RQ3})] Why do developers write to fields of \const{}-qualified objects?
        \\[0.2em]
        \hspace*{1em}
        \begin{minipage}{4.5in}
          (Answer: Buffers and delayed initialization were important reasons,
          but over half the time, we couldn't find any clear reason motivating
          developers' decisions to write through \const{}.)
        \end{minipage}~\\[-0.5em]
\end{enumerate}

Section~\ref{section:conclusion} presents our detailed answers to these questions.

\vspace*{1em}
Our contributions include:
\vspace*{-0.25em}
\begin{itemize}
  \item the design and implementation of a novel dynamic analysis for C++ that
        detects \wstcqd{} variables (both shallow and transitive);
  \item an empirical study of \const{} usage (including \wstcqs{}) on a suite of
        7 C++ benchmarks; and,
  \item based on the empirical study, a novel classification of \wstcqs{} in the
        wild according to a root cause and a set of attributes.
\end{itemize}
