We evaluated our ConstSanitizer tool on 7 C++ software projects, plus 1 C project.
We attempted to choose significant benchmarks using these
guidelines:

\begin{enumerate}
  \item must span a range of application areas: applications and libraries; small,
        medium, and large projects; interactive and non-interactive;
  \item are used by the community: the Google projects are the most popular on
        GitHub; the applications are popular among FOSS users; contributor-group
        sizes vary from a core group to a large community; and,
  \item must extensively use \const{} constructs.
\end{enumerate}

\noindent
A ConstSanitizer report indicates that a write that would not be allowed under
deep immutability occurred through a read-only reference.
Such writes are allowed under C++ semantics.
They are only a departure from the \const{} semantics that we experiment
with (i.e. deep immutability with no casts and no mutable).
Our experiments classify \wstc{} observed in actual \const{}-using programs.
Classifying these writes provides us valuable insight about \const{} usage in
practice, which will guide future work.

Our approach was to modify the project's build system to use our tool and to
disable optimizations.
We then ran the project's test suite, when available, and collected output from
our instrumentation.
Using this output we categorized the writes that we found, assigning root causes
and attributes.
Along with the number of static locations of writes that we found
(\textbf{bolded}), we also report the number of dynamic occurrences of each
write over observed executions.
All else being equal, dynamic counts can help prioritize \wstc{}, with
more-frequent locations to be investigated first.
We refer to these dynamic occurrences of writes as ``occurrences'' in the
sequel.
Table~\ref{table:projects} summarizes our benchmark projects.

\begin{table}[!htb]
  \centering
  \caption{We ran our experiments across 7 C++ (and 1 C) software projects;
           ConstSanitizer introduces a build slowdown of
           1.05$\times$--1.40$\times$ across all projects.}
  \label{table:projects}
  \begin{tabular}{l l r l r}
    & \textbf{Name} & \textbf{Version} & \textbf{Description}
    & \makecell[r]{\textbf{Build} \\ \textbf{Slowdown}}\\
    \midrule
    C++ & Protobuf & 2.6.1 & Serialization framework & 1.40$\times$\\
    C++ & LevelDB & 1.18 & Key/value database & 1.05$\times$\\
    C++ & fish shell & 2.2.0 & UNIX shell & 1.32$\times$\\
    C++ & Mosh (mobile shell) & 1.2.5 & SSH replacement & 1.26$\times$\\
    C++ & LLVM TableGen & 3.7.0 & Domain-specific generator & ---\phantom{\- $\times$} \\
    C++ & Tesseract & 3.04.00 & OCR engine & 1.10$\times$\\
    C++ & Ninja & 1.6.0 & Build system & 1.20$\times$\\
    C & Weston & 1.9.0 & Wayland compositor & 1.28$\times$\\
  \end{tabular}
\end{table}

We recorded relative overhead introduced by our instrumentation with respect to
both building and testing times on the longest-running projects, Protobuf and
LevelDB.
Table~\ref{table:projects} includes build slowdowns induced by our tool, which
ranged between 1.05$\times$ and 1.40$\times$.
Our tool caused a 3.3$\times$ slowdown and 1.3$\times$ slowdown in test
execution times for Protobuf and LevelDB respectively.
The remaining projects were either interactive, or did not have long enough
running test suites to get meaningful results.
We do not any report LLVM TableGen numbers because we built it (with
instrumentation) as part of the LLVM build process and were not able to build
the executable separately.

\subsection{Protobuf}

Protobuf is Google's serializing framework for structured data, consisting of
about 214~000 lines of C++ code.
We analyzed version 2.6.1 of Protobuf by running its test suite, which contains
5 tests.
Table~\ref{table:protobuf-violations} summarizes the Protobuf results.
ConstSanitizer found {\bf 76} static write locations (and 127~644 occurrences).
We describe 5 archetypes for these writes.
An archetype is a group of writes that we judged to be similar; the writes may
happen at different source locations.

\begin{table}[!htb]
  \centering
  \caption{Protobuf shows 5 archetypes for {\bf 76} writes through \const{} resulting in 127~644
           occurrences.}
  \label{table:protobuf-violations}
  \begin{tabular}{l r r c c}
    \textbf{Archetype} & \textbf{Locations} & \textbf{Occurrences}
                     & \textbf{Root Cause} & \textbf{Attributes} \\
    \midrule
    Generator printer & {\bf 7} & 118464  & T & B \& N \& S \\
    Message cache sizes & {\bf 61} & 7158 & M & B \& S \\
    Source code locations & {\bf 4} & 1898 & T & I \\
    Linked list operations & {\bf 2} & 84 & M & I \& S \\
    Generate initialization \\
    \hspace*{2em}  method & {\bf 2} & 40 &  M & D \& N \& S \\
  \end{tabular}
\end{table}

The ``Generator printer'' archetype occurred most often.
Listing~\ref{listing-protobuf-generator-printer} presents a representative
expanded stack trace.
The function at the top of the listing shows the initiation of the write in
\code{Generator}'s \const-qualified \code{Generate} method.
This method calls \code{PrintTopBoilerplate}, passing a pointer to a mutable
\code{io::Printer}.
Then, \code{Printer}'s \code{WriteRaw} method modifies (root cause T) two fields:
\code{buffer\_} and \code{buffer\_size\_}.
These fields are protected by a lock, act as a buffer, and are not visible
outside the class (which is just a printer).
This archetype also includes other \code{Print}-like calls with different source
locations but a common explanation.

\begin{listing}[!htb]
  \caption{Protobuf's Generator class performing transitive \wtc{} to a Printer
           field.}
  \label{listing-protobuf-generator-printer}
  \centering
  \includestandalone{listing/protobuf-generator-printer}
\end{listing}

The ``Generate initialization method'' archetype is related to ``Generator
printer''.
Listing~\ref{listing-protobuf-generator-lazy} shows this archetype.
The \code{printer\_} field was initialized as seen above.
C++ allows this write due to the \texttt{mutable} specifier.
Another field, \code{file\_}, is lazily initialized as well.
Both of these fields are protected by the same lock, and are not externally
visible outside the class.

\begin{listing}[!htb]
  \caption{Protobuf's Generate initialization method performing lazy initialization.}
  \label{listing-protobuf-generator-lazy}
  \centering
  \includestandalone{listing/protobuf-generator-lazy}
\end{listing}

We show an example of the ``linked list operations'' archetype in
Listing~\ref{listing-protobuf-list}.
Here, the \code{depart} method grabs a lock, and uses a pointer with type
\code{linked\_ptr\_internal \const{} *}, so that the \const{} applies to what is
pointed to, not to the pointer.
The method then modifies the \code{next\_} field of a valid object at the point
indicated by the comment.
The root cause here is \code{mutable}: the \code{next\_} field is declared
\code{mutable linked\_ptr\_internal const* next\_}.
This write is a delayed initialization, not visible, and synchronized.

\begin{listing}[!htb]
  \caption{Protobuf using Google test linked list that writes internally.}
  \label{listing-protobuf-list}
  \centering
  \includestandalone{listing/protobuf-list}
\end{listing}

Listing~\ref{listing-protobuf-cache} shows the ``Message cache
sizes'' archetype.
The write is protected by a lock, and is allowed by C++ because the field is
\texttt{mutable}.
However, this write, while involved with caching, is externally visible.
The method \code{void SetCachedSize(int size) const} enables
external code to modify this field through a \const{}
reference to the containing object.

\begin{listing}[!htb]
  \setlength{\belowcaptionskip}{-1em}
  \caption{Protobuf writing to a message's cached size field.}
  \label{listing-protobuf-cache}
  \centering
  \includestandalone{listing/protobuf-cache}
\end{listing}

Listing~\ref{listing-protobuf-locations} shows the ``Source code locations''
archetype.
The \code{mutable\_leading\_comments} method, which includes ``mutable'' in its
name, is not declared as \const{}, and thus allows writes.
Its implementation writes to the \code{location\_} field; we show an example of
a caller which causes such a write.
The \code{location\_} field is externally-visible, so this is a clearly
incorrect externally-visible transitive write; we assign attribute I.

\begin{listing}[!htb]
  \caption{Protobuf writing to a source location object.}
  \label{listing-protobuf-locations}
  \centering
  \includestandalone{listing/protobuf-locations}
\end{listing}

We also found an archetype involving writing data to a message.
This included 133 unique source locations, occurring 14~638 times in total.
However, the code is heavily inlined and the build system appears to overwrite
optimization settings for this subdirectory.
Manual inspection of the code revealed no obvious writes.
We believe this is a result of optimizations causing invalid debugging
information.
We thus omitted this archetype from Table~\ref{table:protobuf-violations}.

\subsection{LevelDB}

LevelDB (1.18) is Google's lightweight key/value database library, consisting of
approximately 18~000 lines of C++ code.
The test suite contains 23 test drivers.
There were 6 archetypes and also {\bf 6} root source locations for these writes.
These locations contributed to 13~792 occurrences over the test drivers.
Table~\ref{table:leveldb-violations} shows a summary of our findings for
LevelDB.

\begin{table}[!htb]
  \centering
  \caption{LevelDB shows writes from {\bf 6} source locations, with 13~792
           occurrences in total.}
  \label{table:leveldb-violations}
  \begin{tabular}{l  r  c  c }
    \textbf{Location} & \textbf{Occurrences} & \textbf{Root Cause}
    & \textbf{Attributes} \\
    \midrule
    \texttt{db/db\_test.cc:40} & 10311 & T & N \& S \\
    \texttt{util/cache.cc:315} & 2841 & T & B \& S\\
    \texttt{db/snapshot.h:54} & 319 & T & I \\
    \texttt{db/snapshot.h:55} & 319 & T & I \\
    \texttt{helpers/memenv/memenv.cc:274} & 1 & T & I \& S \\
    \texttt{util/testutil.h:42} & 1 & T & N \\
  \end{tabular}
\end{table}

Listing~\ref{listing-leveldb-counter} shows the source location that caused
the majority of the occurrences.
This code extends the \code{RandomAccessFile} class to add an atomic counter
field, \code{counter\_}, that tracks the number of read calls.
The root cause is that \code{counter\_} is a pointer and is transitively written
to.
The reason for this write is test controllability: this class is part of the
test infrastructure.
Yet it must override the monitored call (and thus must be \const{}).
This class is meant for testing purposes only, so we concluded the write was not
visible outside the class---the counter is only used in the testing code.

\begin{listing}[!htb]
  \caption{LevelDB write in \texttt{db\_test.cc:40} incrementing counter tracking
           \# of writes to a file.}
  \label{listing-leveldb-counter}
  \centering
  \includestandalone{listing/leveldb-counter}
\end{listing}

Listing~\ref{listing-leveldb-cache} shows a modification to a caching structure
that generates a new identifier.
This cache is a field, \code{block\_cache}, in \code{options}, which is declared
as \code{const Options\&} in \code{Table::Open}.
The root cause is a transitive write, since the code dereferences a field of a
\const{} object to do the write.
This write is protected by a lock and clearly involved in caching.
However, it appears that other code outside of \code{Options} uses this block
cache.

\begin{listing}[!htb]
  \caption{LevelDB write in \texttt{cache.cc:315} creating a new block cache in
           \code{\const{} Options} object.}
  \label{listing-leveldb-cache}
  \centering
  \includestandalone{listing/leveldb-cache}
\end{listing}

Listing~\ref{listing-leveldb-list} shows a modification of a linked list node
accessed through two pointer dereferences.
This corresponds to both \texttt{snapshot.h} locations shown in the table.
This code modifies the pointers obtained from following its own nodes,
performing a transitive write through a \const{} qualifier.
We do not know why the developers declared \texttt{s} as \const{} since it is
also destroyed at the end of the method. In any case, we assigned attribute I.

\begin{listing}[!htb]
  \caption{LevelDB write in \texttt{snapshot.h} deleting a list element and
           updates pointers.}
  \label{listing-leveldb-list}
  \centering
  \includestandalone{listing/leveldb-list}
\end{listing}

Listing~\ref{listing-leveldb-env} shows a \wtc{} in the \code{InMemoryEnv}
class.
As with the cache, the root cause is a transitive write: in the caller,
\code{options} is declared \code{const Options\&}.
Unlike the caching example, this file isn't involved in caching and appears to
be a visible change to \code{options}.
This write is protected by a lock, giving attribute I \& S.

\begin{listing}[!htb]
  \caption{LevelDB write in \texttt{memenv.cc} changing the environment in
           options object.}
  \label{listing-leveldb-env}
  \centering
  \includestandalone{listing/leveldb-env}
\end{listing}

Listing~\ref{listing-leveldb-test} shows the final \wtcq{} that we found for
LevelDB.
The caller location is the same as in Listing~\ref{listing-leveldb-env} above.
In this case, however, the containing class extends \code{InMemoryEnv} and adds
a field to count the number of errors (for testing purposes only).
Therefore we attribute this write as being not visible---it is only used in
tests.

\begin{listing}[!htb]
  \caption{LevelDB write in \texttt{testutil.h} injecting faults into
           the test suite.}
  \label{listing-leveldb-test}
  \centering
  \includestandalone{listing/leveldb-test}
\end{listing}

\subsection{fish shell}

fish shell (2.2.0) is a UNIX shell providing advanced features, consisting of
approximately 48~000 lines of C++ code.
We compiled the project with our tool and executed an instance of the shell.
Our workload launched the shell and immediately exited.
We found writes from {\bf 4} unique source locations for 98 occurrences in total.
All locations are within the \code{exchange} function.
Listing~\ref{listing-fish} shows this function along with a snippet of
\code{\_wgetopt\_internal} that calls \code{exchange}.
The root cause is that the \const{}-qualified \code{argv}
variable gets cast to non-\const{} and then passed to \code{exchange}.
This write shows that the \const{}-qualifier on \code{argv} is incorrect and
should not be included.

\begin{listing}[!htb]
  \caption{fish shell writing to \const{}-qualified \code{argv} object.}
  \label{listing-fish}
  \centering
  \includestandalone{listing/fish}
\end{listing}

\subsection{Mosh (mobile shell)}

Mosh (mobile shell) (1.2.5) is a remote terminal application that is a
replacement for secure shell (SSH), consisting of about 13~000 lines of C++
code.
Our workload was to launch the mosh server and immediately terminate it.
We found \wstc{} at {\bf 8} unique source locations (432 occurrences).
Listing~\ref{listing-mosh} shows one of the writes.
Mosh parsing code sets a flag to indicate completion.
However, the developers declared the parser action as \const{} in the same
method where they modify it.
The root cause is that the variable \code{handled} is declared \code{public
mutable}.
We believe this is an incorrectly \const{} qualified variable.

\begin{listing}[!htb]
  \caption{Mosh handling terminal action with a \wtc{}.}
  \label{listing-mosh}
  \centering
  \includestandalone{listing/mosh}
\end{listing}

\subsection{LLVM TableGen}

We instrumented LLVM's (3.7) TableGen executable, which uses domain-specific
information to generate files with custom backends.
This part of LLVM consists of approximately 34~400 lines of C++ code.
It is primarily used in building LLVM itself.
We added our instrumentation to the build system and observed reports from an
instrumented version of TableGen executing as part of the build process.
LLVM itself is a large body of code with too many \wstcq{} objects to manually
classify.
In TableGen, we found writes from {\bf 3} unique source locations (282
occurrences).

The handling code for DFAs contains some puzzling writes, shown in
Listing~\ref{listing-llvm-dfa}.
The write immediately follows an instantiation of a \code{\const{} State} object.
The \code{State} class itself is only available in a file's translation unit
(not usable outside the file), which may indicate that the \code{State} is not
intended to be widely used.
\code{State} only contains \const{} methods and all of its fields (except one
explicitly declared \const{}) are mutable.
Since all methods are \const{} there is no difference in callable methods
between non-\const{} and \const{}-qualified access.
In addition, since all other fields are mutable, developers are allowed to
re-assign the same fields in a \const{} method as they would in a non-\const{}
method.
Since only one field doesn't have \texttt{mutable}, developers could achieve the
same effect by making all methods non-\const{}, removing all \texttt{mutable}
specifiers on fields, and changing the one field that did not have
\texttt{mutable} to be \const{} qualified.

\begin{listing}[!htb]
  \caption{LLVM DFA code marks \texttt{State} \const{} for no apparent reason.}
  \label{listing-llvm-dfa}
  \centering
  \includestandalone{listing/llvm-dfa}
\end{listing}

The other write is in the code that computes a sub-register index for code
generation.
Listing~\ref{listing-llvm-subreg} shows the containing method.
The root cause here is that the \code{LaneMask} field is mutable.
The write caches the value.
However, this value is not used in any other methods.

\begin{listing}[!htb]
  \caption{LLVM SubReg writes to a \texttt{mutable} field in a \const{} method.}
  \label{listing-llvm-subreg}
  \centering
  \includestandalone{listing/llvm-subreg}
\end{listing}

\subsection{Tesseract}

Tesseract (3.04.00) is an optical character recognition (OCR) engine maintained
by Google, consisting of 147~000 lines of C++ code.
This project does not contain any easy-to-run tests.
We compiled it with our tool and ran it with invalid arguments.
With our limited knowledge of Tesseract's usage, we were not able to cause the
core algorithm to execute.
However, we found a strange write, shown in Listing~\ref{listing-tesseract}.
The root cause is that the \code{used\_} field is mutable.
This write appears to be an incorrect usage of \const{}.
Strangely, however, the comments indicate that is a defensive write against
possible further \wstcqs{}.

\begin{listing}[!htb]
  \caption{Tesseract performs a strange write in its string class.}
  \label{listing-tesseract}
  \centering
  \includestandalone{listing/tesseract}
\end{listing}

\subsection{Ninja}

Ninja (1.6.0) is a build system consisting of approximately 14~900 lines of C++
code.
It includes a modest test suite.
Our tool reports 39 occurrences from calls to the standard library.
All of these warnings have a {\bf single} source location outside the standard
library: \texttt{src/disk\_interface\_test.cc:226:3}.
Listing~\ref{listing-ninja} shows this static source location.
This is a quick hack to run the test suite with the same API as normal clients.
This field stores statistics that are checked in the test suite only.
The field is mutable and not seen outside the test suite, so we give this write
the ``not visible'' attribute.

\begin{listing}[!htb]
  \caption{Ninja \wtc{} in test code.}
  \label{listing-ninja}
  \centering
  \includestandalone{listing/ninja}
\end{listing}

\subsection{Weston}

While we focus on C++ in this work, our technique also works on \const{} in C
programs.
We therefore evaluated it on a C application.
Since this is the sole C project, we omit Weston from the overall table
of results (Table~\ref{table:other}).
Weston (1.9.0) is a reference implementation of a Wayland compositor.
It consists of approximately 85~000 lines of C code and the test suite has 20
tests.
We did not expect to see many writes, as most C standard library functions do
not require \const{} (corresponding C++ library functions usually do), and
also due to annoyances in using \const{} in C, which we describe below.
However, even with a small test suite, we found {\bf 4} unique source locations
for writes (accounting for 115 occurrences).

All of the \wstc{} are transitive and came from parsing code.
The argument option parser accounts for {\bf 3} locations.
Listing~\ref{listing-weston-option} shows a write in the parser.
Function \code{handle\_option} does not modify the pointer value of
\code{option} but modifies its transitive data field.
This does not change any data stored in the \code{weston\_option} structure,
maintaining bitwise \const{}-ness.
The \code{data} field's type is \code{void *} and the cast does not remove
\const{}.
Based on the function name, one might expect a write to the \code{data} field,
not its pointee.

\begin{listing}[!htb]
  \caption{Weston option parser modifying its \const{} option argument.}
  \label{listing-weston-option}
  \centering
  \includestandalone{listing/weston-option}
\end{listing}

The final location was in the configuration file parsing code.
Listing~\ref{listing-weston-config} shows the
\code{weston\_config\_section\_get\_uint} function dereferencing and modifying
the \code{value} argument passed in from a field of a \const{} struct.
As above, based on the function naming, one would expect any writes to happen
through the \code{dest} pointer.
This write does not modify any data stored in the \code{config\_command}
structure and maintains bitwise \const{}-ness as well.

\begin{listing}[!htb]
  \caption{Weston config parser writing to its \code{value} argument.}
  \label{listing-weston-config}
  \centering
  \includestandalone{listing/weston-config}
\end{listing}

We made an observation as to why \const{} may be unattractive to C developers:
there is no clean way to initialize a structure analogous to C++
constructors/destructors.
A popular C idiom is to assign constructor-like functions signatures like
\code{rec\_init(struct rec *r)}.
This signature prevents initialization without casting: \code{\const{} struct
rec r; rec\_init(\&r)} is illegal.
However, it is cumbersome to always cast for constructor-like calls.
One could change the signature of the function to \code{rec\_init(const struct
rec *r)} and perform the cast in the function.
However, that function would violate shallow immutability---it writes to fields
as it initializes them.
Using \const{} in C appears to require developers to ignore the casting away of
\const{} qualifiers for constructor-like functions.

\subsection{Summary}
\label{section:results-summary}

Table~\ref{table:other} summarizes the \wstcqs{} from benchmarks other than
Protobuf and LevelDB.
Across the 7 C++ projects we instrumented and ran, we observed 17 unique
archetypes across a total of 142~288 dynamic occurrences.
We manually divided these archetypes into 17 classifications.
The root causes were evenly split between writes through mutable fields and
transitive writes (8 of each) with one \wtc{} due to casting.
Valid attributes were mostly with-synchronization and because the write was not
visible (7 and 6 respectively).
The other valid attributes, writing to a buffer/cache and delayed
initialization, occurred 4 times and 1 time respectively.
The majority attribute, in 9 cases, was that the write was incorrect and
violated intuitive notions of what \const{} should mean.
We reported our results to developers. 
Within a few days the developers simply removed incorrect \const{} qualifiers
in both fish and Mosh.

\begin{table}[!htb]
  \centering
  \caption{\Wstcqs{} in our other benchmark programs were mainly incorrect uses
           of \const{}.}
  \label{table:other}
  \small
  \begin{tabular}{l l r c c}
    \textbf{Project} & \textbf{Location} & \textbf{Occurrences}
                     & \textbf{Root Cause} & \textbf{Attributes} \\
    \midrule
    fish shell & \texttt{wgetopt.cpp} & 98 & C & I \\
    Mosh & \texttt{terminal.cc} & 432 & M & I \\
    LLVM & \texttt{DFAPacketizerEmiter.cpp} & 112  & M & I \\
    ~~TableGen    & \texttt{CodeGenRegisters.cpp} & 170  & M & B \& N \\
    Tesseract & \texttt{ccutil/strngs.cpp} & 1 & M & I \\
    Ninja & \texttt{disk\_interface\_test.cpp} & 39 &  M & N \\
  \end{tabular}
\end{table}

We found 3 projects (LevelDB, Mosh, and Ninja) had writes that only occurred in
tests.
The \wstc{} we found were in testing code; writes were to counters only present
in test environments.
In Mosh, the fact that the writes were only for test purposes was not
immediately obvious.
However, discussions with the developers revealed that the \code{handled}
variable was only used for debugging.
All of these \wstc{} are related to test controllability, suggesting that this
idiom should be supported directly in the programming language.
