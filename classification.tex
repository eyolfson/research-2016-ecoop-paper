One of our contributions is a careful analysis of the \const{} usages detected
by our ConstSanitizer dynamic analysis tool.
We propose a classification for \wstcqs{} along 2 axes.
We manually assigned each write 1) a single cause, from a set of common root
causes; and 2) a set of additional attributes. This classification distills our
empirical observations about \const{} use in practice.

\begin{table}[ht!]
  \caption{Root causes of writes through \const{} and our symbols for these causes.}
  \label{table:root-cause-reference}
  \centering
  \begin{tabular}{l c}
    \textbf{Root Cause} & \textbf{Symbol} \\
    \midrule
    Write to mutable field         & M \\
    Transitive write               & T \\
    Write after casting a \const{} qualifier away & C \\
  \end{tabular}
\end{table}

\noindent Table~\ref{table:root-cause-reference} lists all of the root causes
for \wstc, along with a one-letter abbreviation that
we will use in Section~\ref{section:results}'s tables.
ConstSanitizer detects such writes and reports them to the user.
The causes are:

\begin{itemize}
  \item mutable field (M): the program writes to a \texttt{mutable}-labelled
        field of a \const{} object.
        \begin{center}
          \includestandalone{listing/mutable-field}
        \end{center}
        \texttt{mutable} permits method \texttt{mutator()} to
        write to field \texttt{x} even though it is a \const{} method, which
        would ordinarily prevent (at compile-time) writes to fields of the
        \texttt{this} object.

  \item transitive write (T): the program writes through a field of a \const{}
        object.
        \begin{center}
          \includestandalone{listing/transitive-write}
        \end{center}
        \const{}-qualified method \texttt{transitiveWrite()} writes to
        field \texttt{x} of the \texttt{this} object.
        While the \const{} qualifier prevents mutation of the \texttt{x} field,
        it does not prevent transitive writes of the memory pointed to by
        \texttt{x}.

  \item casting away const (C): the program writes through a pointer which has
        previously been \const{} but whose \const{}-ness has been cast away
        using a \code{const\_cast} or C-style cast.
        \begin{center}
          \includestandalone{listing/const-cast}
        \end{center}
        The write in \texttt{writeToArg()} mutates the value pointed-to by
        \texttt{x} while \texttt{x} is \const{}-qualified.
        ConstSanitizer reports \wstcqs{} whose \const{}-ness has been
        cast away, using the \const{}-ness of the most recent
        declared type for the value.
\end{itemize}

\vspace*{-1em}
\begin{table}[ht!]
  \caption{Observed common attributes of writes through \const{} and corresponding
           symbols.}
  \label{table:classification-reference}
  \centering
  \begin{tabular}{l c}
    \textbf{Attribute} & \textbf{Symbol} \\
    \midrule
    Write is synchronized & S \\
    Write is not visible  & N \\
    Write is to a buffer/cache & B \\
    Write is delayed initialization & D \\
    Write is incorrect & I \\
  \end{tabular}
\end{table}

Table~\ref{table:classification-reference} summarizes our attributes for \wstcqs.
We assigned attributes to writes based on our understanding of the code.
Writes may have multiple attributes; for instance, a
write in our Protobuf benchmark is B \& N \& S.
The attributes are:

\begin{itemize}
  \item synchronized (S): indicates that the write is always protected by a
        lock.
        This attribute is often required under the C++11 standard: all types
        that are shared between threads and that may be used with the standard
        library must be either bitwise const, which is clearly not the case when
        we witness a write, or else protected against concurrent
        accesses~\cite{sutter13:_gotw_solut}.
        The following example, from Protobuf, is synchronized using Google mutex
        primitives.
        \begin{center}
          \includestandalone{listing/synchronized}
        \end{center}
  \item not visible (N): indicates that the result of the write is never
        externally visible (e.g. private and with no accessor methods; may be
        accessed in the same translation unit).
        Often occurs in the context of testing-related counters.
        \begin{center}
          \includestandalone{listing/not-visible}
        \end{center}
  \item to a buffer/cache (B): indicates that the write is of a derived value
        which can be computed from other currently-available state.
        Such writes are often optimizations.
        \begin{center}
          \includestandalone{listing/buffer}
        \end{center}
  \item delayed initialization (D): indicates that the write initializes state
        not initialized in the constructor or its transitive
        callees.
        Writes with this attribute could have also occurred
        in the constructor, but the written value was not yet
        available.
        Failure to call a delayed initialization method would lead to undesired
        behaviours (or lack of desired behaviours).
        \begin{center}
          \includestandalone{listing/delayed-initialization}
        \end{center}
  \item incorrect (I): indicates that the write appeared to violate the
        \const{}-ness of the object.
\end{itemize}

Note that S/N/B/D writes are not necessarily errors and do not necessarily
violate immutability properties.
We thus chose the word ``attribute'' to suggest that S/N/B/D indicate an
incidental property of a \wtc.
If the code containing the writes is properly written, an object with an
S/N/B/D \wtc{} can still appear to be immutable to the client, assuming all
references to that object are read-only.
A write with attribute I, however, is a client-visible violation of \const{}.
