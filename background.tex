As discussed earlier, the C++ \const{} keyword allows programmers to declare,
in some sense, that a value should not change.
In this section, we explain the specific guarantees that C++ provides, and we
present the deep immutability variant of these guarantees that we verify.

\subparagraph*{Meaning of \const{} on C++ primitive and pointer types.}

The meaning of \const{} in C++ is an extension of its meaning in C.
We start by describing the common meaning of \const{} across C and C++, applying
to primitive and pointer types.
In these languages, the \const{}-ness of a memory location depends on the
qualifiers of the variable through which the location is accessed; our
motivating example illustrated a change in \const{}-ness through a cast, in
function \texttt{evil()}, that is allowed by both C and C++.

Developers may \const{}-qualify primitive types such as \code{\texttt{int}},
resulting in immutable object types like \code{\const{} \texttt{int}}.
When variable $v$ has primitive \const{}-qualified type, the C++ type system
prevents developers from assigning to $v$ after its definition; i.e. it
prevents writes to $v$.
In this case, \const{} behaves like \texttt{final} in Java, which also
prevents re-assignment.

For pointer types, such as \code{\texttt{int *}}, developers may
\const{}-qualify both the pointee and pointer type.
The \const{} qualifier applies to the type directly to the left of it; if there
is nothing to the left, then \const{} applies to the right.
If a variable has type \code{\texttt{int *}\const{}}, developers may not change
the address value of the variable (where it points to), but they may dereference
the location and change the value it points to.
A different type is \code{\texttt{int} \const{} \texttt{*}} (also known as
\code{\const{} \texttt{int} \texttt{*}}), which allows the address value to
change, but not the value pointed-to by the variable.
This type represents a read-only reference.
The \const{} qualifier may also apply to both the pointer and pointee types
(\code{\texttt{int} \const{} \texttt{*} \const{}}), which prevents writes to
both the address value and the value pointed-to.
If all pointers to a value are read-only references, then the value pointed-to
is immutable.
C++ references can be thought of as \const{}-qualified pointers; a developer may
not write to the reference address value.
However, in contrast with pointers, developers cannot cast away reference
address value \const{}-ness and re-assign the address value; this property is
enforced by the language.

\begin{listing}[!htb]
  \caption{The \const{} qualifier may apply to C++ member functions.}
  \label{listing-const-background}
  \centering
  \includestandalone{listing/const-background}
\end{listing}

\subparagraph*{Meaning of \const{} on C++ object types.}
We continue by exploring the meaning of \const{} in C++-specific contexts.
When a C++ object type is \const-qualified,
the developer may only call member functions declared with a
\const{} qualifier.

\const{}-qualifying a member function has two effects.
First, \const{}-qualifying a member function allows it to be called on a
\const{}-qualified receiver object.
Furthermore, inside the function, the type qualifiers of the receiver object's
fields are treated as \const{}.

Conceptually, each C++ class provides two
interfaces: the \const{}-qualified interface and the non-\const{} qualified
interface.
A \const{}-qualified reference is meant to be a read-only
reference, although C++ enforces no guarantees.
One of our goals is to evaluate whether read-only guarantees hold in practice.
When an object has non-\const{} type, then the developer may call all methods on
that object\footnote{There is a small exception: on a non-\const{} object, the
developer cannot call \const{}-qualified methods that are hidden due to
overloading by a non-\const{}-qualified method of the same signature.}.
On the other hand, when an object has \const{} type, then the developer may only
call methods on that object that are \const{}-qualified.

Consider class \texttt{Pointish}, defined in
Listing~\ref{listing-const-background}.
As written, the developer could call all methods on a non-\const{}-qualified
object of type \texttt{Pointish}.
On the other hand, the developer may only call the \texttt{getX()} method on a
\const{}-qualified \texttt{Pointish} object.

The second effect of \const{}-qualifying a function changes the type qualifiers of
fields inside the function.
In our example, field \texttt{x} becomes \texttt{int}
\const{} within \const{}-qualified member functions.
The compiler successfully compiles \texttt{getX()}, since there are no
writes to \texttt{x} or \texttt{y}.
But, if \texttt{setX()} was \const{}-qualified, the compiler would refuse to
compile the code, since the type of \texttt{x} would be treated as \const{}
\texttt{int} and \texttt{setX()} contains a write to that variable.

In C++, without using \const{} escape hatches, developers may re-assign
fields in non-\const{} qualified methods and may not re-assign fields in
\const{}-qualified methods.
In all methods, developers are permitted to mutate state outside of
re-assignment (through references or pointers).
This type of immutability is referred to as \emph{shallow immutability}.

A C++ \const{}-qualified stack/global object would be considered a shallow
\emph{immutable object}.
That is, without escape hatches, developers cannot create non-\const{}
references (including through pointers) to such a \const{}-qualified object.
However, as we discuss next, developers may indeed remove the \const{} qualifier
on references to the \const{}-qualified object.
Therefore, C++ does not strongly enforce the concept of an immutable object.

\subparagraph*{Working around \const{} restrictions.}
Practical type systems appear to require escape hatches.
C++'s escape hatches for \const{} include casting (the sole
escape hatch in C) and \texttt{mutable}.
Also, C++ \const{} does not specify deep immutability.
ConstSanitizer dynamically observes executions to
monitor uses of escape hatches and deep immutability.

Most type systems permit casting between types.
C-style casts (\code{(const A)a}) and C++ \texttt{const\_cast}s can
add or remove \const{} qualifiers.
\const{} manipulation may also occur through unions and
\texttt{reinterpret\_cast}s.
ConstSanitizer ignores casts, instead using the declared type of
a variable or function argument.
When there is a mismatch between variable and function argument types, we
persist all \const{} information.

For \const{} member functions, ``\texttt{mutable}'' instructs the
compiler to not add the implicit \const{} type qualifier otherwise imposed on
fields inside those functions.
In Listing~\ref{listing-const-background}, if \texttt{x} were instead
declared as \code{\texttt{mutable int x}}, then \texttt{setX()} could be
\const{}-qualified and still compile.
ConstSanitizer would then report the write to field \texttt{x} whenever
\texttt{setX()} was called on a \const{} receiver object, as we consider the
write to be a breach of that object's immutability.

ConstSanitizer goes beyond C++'s guarantees with respect to transitivity.
Consider again Listing~\ref{listing-const-background}.
Field \texttt{y} has type \code{int *}; within a \const{}-qualified member
function, its type is \code{int *const}.
C++ therefore prevents writes to \texttt{y} inside a \const{}-qualified member
function.
However, it does not prevent writes to \texttt{*y} without further explicit
markup (i.e. \code{int const *const}).
We call writes to locations like \texttt{*y} \emph{transitive writes}.

C++ only guarantees shallow immutability---that is, field values
directly stored in a \const{}-qualified class do not change.
If a field has pointer type, C++ ensures that the pointer value does not change,
but does not guarantee anything about the value pointed to.
ConstSanitizer verifies \emph{deep immutability} through transitive writes, and
enables us to answer the empirical question of whether extant programs preserve
deep immutability or not.
